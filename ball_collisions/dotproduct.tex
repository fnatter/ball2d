%% -*- latex -*-
\documentclass[math,plainoldenumerate,afour]{homework}

\usepackage[dvips]{graphics}
\usepackage{psfrag}
\usepackage[T1]{fontenc}
\usepackage[cdot]{SIunits}

\begin{document}

\pagestyle{plain}
\makebragbox{Dot product: axioms, geometr. explanation and applications}{1999}

\section{Axioms}

\begin{description}
\item[bilinear]
  $\vec{a}\cdot(\vec{b}+\vec{c})=\vec{a}\cdot\vec{b}+\vec{a}\cdot\vec{c}$
\item[?]
  $\vec{b}\cdot(c\cdot\vec{a})=\vec{a}\cdot(c\cdot\vec{b})$
\item[symmetric]
  $\vec{a}\cdot\vec{b}=\vec{b}\cdot\vec{a}$
\item[positive]
  $\vec{x}^2>0$ ($\vec{x}\not=\vec{0}$)
\end{description}

Furthermore, the magnitude of vectors is defined using the dot product:
$|\vec{x}|^2=\vec{x}^2$ and, because $\vec{x}^2>0$:
$|\vec{x}|=\sqrt{\vec{x}^2}$.

\section{Geometric explanation}
\label{sec:dotproductgeometry}

\begin{figure}[ht]
  \Large
  \psfrag{veca}{$\vec{a}$}
  \psfrag{vecb}{$\vec{b}$}
  \psfrag{vecbma}{$\vec{b}-\vec{a}$}
  \psfrag{x1}{$x_1$}
  \psfrag{x2}{$x_2$}
  \psfrag{x3}{$x_3$}
  \psfrag{theta}{$\theta$}
  \begin{center}
    \includegraphics*{dotproduct.eps}
  \end{center}
  \caption{geometric explanation of the dot product}
  \label{fig:dotproductgeometry}
\end{figure}

According to the law of cosine (figure \ref{fig:dotproductgeometry}):
\begin{multline*}
  |\vec{b}-\vec{a}|^2=|\vec{a}|^2+|\vec{b}|^2-2|\vec{a}||\vec{b}|\cos(\theta)
  \Leftrightarrow
  |\vec{a}|^2+|\vec{b}|^2-|\vec{b}-\vec{a}|^2=2|\vec{a}||\vec{b}|\cos(\theta)
  \Leftrightarrow \\
  a_1^2+a_2^2+a_3^2+b_1^2+b_2^2+b_3^2
  -a_1^2+2a_1b_1-b_1^2-a_2^2+2a_2b_2-b_2^2-a_3^2+2a_3b_3-b_3^2 = \\
  2|\vec{a}||\vec{b}|\cos(\theta)
  \Leftrightarrow
  a_1b_1+a_2b_2+a_3b_3 = |\vec{a}||\vec{b}|\cos(\theta)
\end{multline*}

\section{Applications}

\subsection{Proof of pythagorean theorem}

$\vec{c}=\vec{a}+\vec{b}$ ist die Hypothenuse, 
$\vec{b}\cdot\vec{a}=0$:
\texttt{Statement:} $\vec{a}^2+\vec{b}^2=\vec{c}^2$

$\vec{c}=\vec{a}+\vec{b}$ ist true, now square this equation:
$\vec{c}^2=\vec{a}^2+2\vec{a}\vec{b}+\vec{c}^2
\Leftrightarrow
\vec{c}^2=\vec{a}^2+\vec{b}^2$\TODO{is this the legitimate proof ?}

\subsection{Scalar projection and 2-dimensional ball-collision}

\newcommand*{\vone}{\ensuremath{\vec{v_1}}}
\newcommand*{\vtwo}{\ensuremath{\vec{v_2}}}
\newcommand*{\vonei}{\ensuremath{\vec{v_{1_i}}}}
\newcommand*{\vtwoi}{\ensuremath{\vec{v_{2_i}}}}
\newcommand*{\los}{\ensuremath{\vec{C_1C_2}}}

\begin{figure}[ht]
  \Large
  \psfrag{vecv1bef}{$\mathbf{\vonei}$}
  \psfrag{vecv2bef}{$\mathbf{\vtwoi}$}
  \psfrag{vecv1}{$\mathbf{\vone}$}
  \psfrag{vecv2}{$\mathbf{\vtwo}$}
  \psfrag{x}{$x$}
  \psfrag{y}{$y$}
  \begin{center}
    \includegraphics*{Ballkollision.eps}
  \end{center}
  \caption{Example of an elastic 2-dimensional Ballcollision as an
    application of scalar projection}
  \label{fig:ballkollision}
\end{figure}

I won't start out explaining 1-dimensional collisions explicitly, because
2D-collisions can be reduced to 1-dimensional collisions using scalar
projections of the velocity-vectors\cite{mcasco}.

$C_1(-10,0)$ is the center (of mass) of ball 1, $C_2(11/17)$ that of ball
2.  The pre-collision velocity of ball 1 is
$\vonei=\flatvectortwo{10}{5}\metrepersecondnp$ and that of ball 2 is
$\vtwoi=\flatvectortwo{-15}{0}\metrepersecondnp$. The mass of ball 1 is
$2.2\kilogram$ and ball 2 weighs $1.8\kilogram$.

In order to find the momentum of each ball after the collision, you have to
find the amount of each ball's velocity which acts towards the center of
mass of the other ball.  And here is where the dot product comes in handy:
We are looking for the component of \vonei or \vtwoi that is parallel to
the connection of the centers of mass (``line of sight''), given by $\los =
\flatvectortwo{21}{17}\metre$.  This is equivalent to the
\underline{scalar projection of $\vonei$ and $\vtwoi$ on $\los$}.

%% explain scalar projection
The scalar projection of $\vec{a}$ onto $\vec{n}$ is defined as
$\scalarprojectionof{\vec{a}}{\vec{n}}$ or $\vec{a}\cdot\vec{n}^0$. This
can be understood easily by looking at the geometric definition of the dot
product (section \ref{sec:dotproductgeometry}).  Note that the
``axis''-vector (the one that will be projected on) must be normalized.
Thus we must find the unit-vector of $\los$ ($\los^0$) which is
$\frac{los}{\sqrt{730}}$ (the length should be $r_1 + r_2$, though).

%% calculate scalar projection
I define $v_{1_i}$ as the part of the velocity of ball 1 acting
\emph{in line of sight} (\textbf{not $\mathbf{|\vonei|}$!}): \\
$v_{1_i} = \scalarprojectionof{\vonei}{\los^0} =
\flatvectortwo{10}{5}\cdot\flatvectortwo{21}{17}\frac{1}{\sqrt{730}} =
\frac{295}{\sqrt{730}} = 10.9184\metrepersecondnp$.
Similarly, $v_{2_i}$ is the velocity of ball 2 in line of sight:
$v_{2_i} = \scalarprojectionof{\vtwoi}{\los^0} =
\flatvectortwo{-15}{0}\cdot\flatvectortwo{21}{17}\frac{1}{\sqrt{730}} =
\frac{-315}{\sqrt{730}} = -11.6587\metrepersecondnp$ (the negative
value is okay since this is a projection).

Having done this, we can now consider the collision as being
one-dimensional if we choose the line of sight ($\los$) to be the one axis
(which we just did). This implies that we need to express the
post-collision velocities in this 1D-space before we can translate the
results back to 2-space. Thus, I define $v_1$ and $v_2$ the post-collision
velocity- components in line of sight (and again: \textbf{not
  $\mathbf{|\vone|/|\vtwo|}$}).

In the one-dimensional case, two conditions hold:
\begin{equation}
  \begin{split}
    m_1\cdot v_{1_i}+m_2\cdot v_{2_i}=
    m_1\cdot v_1+m_2\cdot v_2
    \Leftrightarrow \\
    m_1\cdot\left(v_{1_i}-v_1\right)=
    m_2\cdot\left(v_2-v_{2_i}\right)
    \quad\text{(conservation of linear momentum)}
    \label{eq:conservationoflinmom}
  \end{split}
\end{equation}

\begin{equation}
  \begin{split}
    \frac{1}{2}\cdot m_1\cdot v_{1_i}^2+\frac{1}{2}\cdot m_2\cdot v_{2_i}^2
    =\frac{1}{2}\cdot m_1\cdot v_1^2+\frac{1}{2}\cdot m_2\cdot v_2^2
    \Leftrightarrow \\
    m_1\cdot\left(v_{1_i}^2-v_1^2\right)=
    m_2\cdot\left(v_2^2-v_{2_i}^2\right)
    \quad\text{(conservation of kinetic energy)} 
    \label{eq:conservationofkinenergy}
  \end{split}
\end{equation}

(\ref{eq:conservationoflinmom}) and (\ref{eq:conservationofkinenergy})
yield a linear system with two equations and two unknowns ($v_1$ and $v_2$):
\begin{multline*}
  m_1 = m_2\frac{v_2-v_{2_i}}{v_{1_i}-v_1}
  \logicaland
  m_1 = m_2\frac{v_2^2-v_{2_i}^2}{v_{1_i}^2-v_1^2}
  \Leftrightarrow
  \frac{v_2-v_{2_i}}{v_{1_i}-v_1}
  = \frac{(v_2-v_{2_i})\cdot(v_2+v_{2_i})}{(v_{1_i}-v_1)\cdot(v_{1_i}+v_1)}
  \Leftrightarrow \\
  v_{1_i}+v_1 = v_{2_i}+v_2
  \Leftrightarrow
  v_1(v_2) = v_2+v_{2_i}-v_{1_i}\logicaland
  v_2(v_1) = v_1+v_{1_i}-v_{2_i}
\end{multline*}

Note that I assumed that $v_{1_i}-v_1\not=0$, $v_{1_i}^2-v_1^2 =
(v_{1_i}-v_1)\cdot(v_{1_i}+v_1)\not=0$ and $v_2-v_{2_i}\not = 0$ (which is
okay, because no collision would take place if the pre- and post-collision
velocities were identical) and $m_2\not = 0$.

Plug $v_1(v_2)$ into (\ref{eq:conservationoflinmom}) and solve for $v_2$:
\begin{equation*}
  \begin{split}
    m_1\cdot(v_2+v_{2_i}-2v_{1_i})=m_2\cdot(v_{2_i}-v_2)
    \Leftrightarrow
    m_1v_2 +m_1v_{2_i}-2m_1v_{1_i}=m_2v_{2_i}-m_2v_2
    \Leftrightarrow \\
    v_2\cdot(m_1+m_2)=v_{2_i}\cdot(m_2-m_1)+2m_1v_{1_i}
  \end{split}
\end{equation*}

Repeating the same procedure for $v_1$, you get:
\begin{gather}
  \label{eq:vone}
  v_1(v_{1_i},v_{2_i},m_1,m_2)=
  v_{1_i}\frac{m_1-m_2}{m_1+m_2}+v_{2_i}\frac{2m_2}{m_1+m_2} \\
  \label{eq:vtwo}
  v_2(v_{1_i},v_{2_i},m_1,m_2)=
  v_{2_i}\frac{m_2-m_1}{m_1+m_2}+v_{1_i}\frac{2m_1}{m_1+m_2}
\end{gather}

At this point, a special case becomes obvious: What if both balls
have equal mass (i.e pool, if you neglect non-linear momentum and friction) ?
In that case, $v_1=v_{2_i}$ and $v_2=v_{1_i}$.
%% ... ?

For our example, plugging into (\ref{eq:vone}) results in: \\
$v_1=\frac{295}{\sqrt{730}}\metrepersecondnp\cdot\frac{1}{10}
-\frac{315}{\sqrt{730}}\metrepersecondnp\cdot\frac{9}{10}
=\frac{-254}{\sqrt{730}}\metrepersecondnp=-9.40096\metrepersecondnp$
and from (\ref{eq:vtwo}): \\
$v_2=-\frac{315}{\sqrt{730}}\metrepersecondnp\cdot-\frac{1}{10}+
\frac{295}{\sqrt{730}}\metrepersecondnp\cdot 1.1
=\frac{356}{\sqrt{730}}\metrepersecondnp=13.1762\metrepersecondnp$
(again, negative signs are valid, because $v_1$ and $v_2$ only 
represent the ``intensity'' of each ball's velocity in line of sight).

In a 1D-collision, I'd be done now: $v_1$ would be the resulting velocity
of ball 1 and $v_2$ that of ball 2. But the 1D-case is a special case
in this respect (see below); for the two- or three-dimensional case,
you have to find the difference between pre- and post-collision velocity
in 1D (``change in line of sight -velocity''), project that onto $\los$
to get the 2-dimensional change in velocity, and add to the pre-collision
velocity to get the final, post-collision velocity (in the 1D-case, this
step is unnecessary because $v_{1_i}+\triangle v_1=v_{1_i}+v_1-v_{1_i}=v_1$).

$\los^0 = \cvectortwo{21\cdot {730}^{-0.5}}{17\cdot {730}^{-0.5}}
=\flatvectortwo{0.7772}{0.6292}$.
Consequently, the change in line of sight-velocity of ball 1 projected
on $\los$ is:
$\overrightarrow{\triangle v_1}=\los^0\cdot(v_1-v_{1_i})
=\flatvectortwo{-15.79221}{-12.78494}$,
and therefore the post-collision velocity of ball 1 is:
$\vone = \vonei + \overrightarrow{\triangle v_1}
=\flatvectortwo{-5.79221}{-7.78494}\metrepersecondnp$.
The post-collision velocity of ball 2 can be calculated in the same way:
$\overrightarrow{\triangle v_2}=\los^0\cdot(v_2-v_{2_i})
=\flatvectortwo{19.30168}{15.62612}$
und $\vec{v_2}=\vec{v_{2_i}}+\overrightarrow{\triangle v_2}
=\flatvectortwo{4.30168}{15.62612}\metrepersecondnp$.

\begin{thebibliography}{1}
\bibitem{mcasco} \texttt{http://www.mcasco.com/p1lmc.html}
\end{thebibliography}

\end{document}
