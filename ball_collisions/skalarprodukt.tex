\documentclass[nativeinput,german,math,plainoldenumerate,afour]{homework}

\usepackage[dvips]{graphics}
\usepackage{psfrag}
\usepackage[T1]{fontenc}

\begin{document}

                                % TODO:
                                % - use comp_x y
                                % - translate to english

\makebragbox{Skalarprodukt: Axiome, geometr. Erkl�rung und Anwendungen}
{1999}

\section{Axiome}

\begin{description}
\item[Distributivgesetz] 
  $\vec{a}\cdot(\vec{b}+\vec{c})=\vec{a}\cdot\vec{b}+\vec{a}\cdot\vec{c}$
\item[Gemischtes Assoziativgesetz]
  $\vec{b}\cdot(c\cdot\vec{a})=\vec{a}\cdot(c\cdot\vec{b})$
\item[Kommutativgesetz] $\vec{a}\cdot\vec{b}=\vec{b}\cdot\vec{a}$
\item[positiv definit] $\vec{x}^2>0$ ($\vec{x}\not=\vec{0}$)
\end{description}

Ausserdem wird dadurch die L\"ange eines Vektors definiert:
$|\vec{x}|^2=\vec{x}^2$ und wegen $\vec{x}^2>0$ auch
$|\vec{x}|=\sqrt{\vec{x}^2}$.

\section{geometrische Einf\"uhrung}

\begin{figure}[ht]
  \Large
  \psfrag{veca}{$\vec{a}$}
  \psfrag{vecb}{$\vec{b}$}
  \psfrag{vecbma}{$\vec{b}-\vec{a}$}
  \psfrag{x1}{$x_1$}
  \psfrag{x2}{$x_2$}
  \psfrag{x3}{$x_3$}
  \psfrag{theta}{$\theta$}
  \begin{center}
    \includegraphics*{dotproduct.eps}
  \end{center}
  \caption{geometrische Einf\"uhrung des Skalarproduktes}
  \label{fig:dotproductgeometry}
\end{figure}

Nach dem Kosinussatz gilt (siehe Abbildung \ref{fig:dotproductgeometry}):

\begin{multline*}
  |\vec{b}-\vec{a}|^2=|\vec{a}|^2+|\vec{b}|^2-2|\vec{a}||\vec{b}|\cos(\theta)
  \Leftrightarrow
  |\vec{a}|^2+|\vec{b}|^2-|\vec{b}-\vec{a}|^2=2|\vec{a}||\vec{b}|\cos(\theta)
  \Leftrightarrow \\
  a_1^2+a_2^2+a_3^2+b_1^2+b_2^2+b_3^2
  -a_1^2+2a_1b_1-b_1^2-a_2^2+2a_2b_2-b_2^2-a_3^2+2a_3b_3-b_3^2= \\
  2|\vec{a}||\vec{b}|\cos(\theta)
  \Leftrightarrow
  a_1b_1+a_2b_2+a_3b_3=|\vec{a}||\vec{b}|\cos(\theta)
\end{multline*}

\section{Anwendungen}

\subsection{Pythagoras-Beweis}

$\vec{c}=\vec{a}+\vec{b}$ ist die Hypothenuse, 
$\vec{b}\cdot\vec{a}=0$:
\texttt{Behauptung:} $\vec{a}^2+\vec{b}^2=\vec{c}^2$

$\vec{c}=\vec{a}+\vec{b}$ ist wahr, quadriere diese Gleichung:
$\vec{c}^2=\vec{a}^2+2\vec{a}\vec{b}+\vec{c}^2
\Leftrightarrow
\vec{c}^2=\vec{a}^2+\vec{b}^2$ -- geht das mit rechten Dingen zu ?

\subsection{Skalarprojektion und Ballkollisionen}

\begin{figure}[ht]
  \Large
  \psfrag{vecv1bef}{$\mathbf{\vec{v_{1_i}}}$}
  \psfrag{vecv2bef}{$\mathbf{\vec{v_{2_i}}}$}
  \psfrag{vecv1}{$\mathbf{\vec{v_1}}$}
  \psfrag{vecv2}{$\mathbf{\vec{v_2}}$}
  \psfrag{x}{$x$}
  \psfrag{y}{$y$}
  \begin{center}
    \includegraphics*{Ballkollision.eps}
  \end{center}
  \caption{elastische 2-dimensionale Ballkollision
    als Anwendung der Skalarprojektion}
  \label{fig:ballkollision}
\end{figure}

Abbildung \ref{fig:ballkollision}: Beispiel f\"ur eine elastische
Ballkollision in der Ebene.

Ich fange hier nicht mit einem Beispiel von 1-dimensionalen Kollisionen an,
weil man -- so wie ich es hier mache -- eine 2D-Kollision mittels
Skalarprojektion ohnehin mit den Methoden f\"ur den 1D-Fall berechnet.

$M_1(-10,0)$ ist der Mittelpunkt (=Massenschwerpunkt)
von Ball 1, $M_2(11/17)$ der von Ball 2,
$\vec{v_{1_i}}=\cvectortwo{10}{5}\frac{m}{s}$ ist
die Geschwindigkeit von Ball 1 vor der Kollision und
$\vec{v_{2_i}}=\cvectortwo{-15}{0}\frac{m}{s}$
die von Ball 2. Die Masse von Ball 1 ist $2.2$ kg und
Ball 2 wiegt $1.8$ kg.

Um herauszufinden, welchen Impuls die B\"alle nach der Kollision haben
werden, muss man jeweils den Anteil der Geschwindigkeit eines Balles finden,
der direkt auf den anderen Ball wirkt. Also sucht man den Anteil von
$\vec{v_{1_i}}$ bzw. $\vec{v_{2_i}}$, der parallel zur Verbindung
der beiden Massenschwerpunkte
($\overrightarrow{M_1M_2}=\cvectortwo{21}{17}$, \glqq Sichtlinie\grqq) ist.
Das entspricht der \emph{Skalarprojektion von
  $\vec{v_{1_i}}$ bzw. $\vec{v_{2_i}}$ auf $\overrightarrow{M_1M_2}$}.
Daf\"ur braucht man zun\"achst die L\"ange von $\overrightarrow{M_1M_2}$,
welche gleich der Summe der beiden Radien
oder $\sqrt{21^2+17^2}=\sqrt{730}$ ist.

%% calculate scalar projection
Ich definiere $v_{1_i}$ als den Geschwindigkeitsbetrag von Ball 1
\emph{in Sichtline} (\textbf{nicht als $\mathbf{|\vec{v_{1_i}}|}$!}): \\
$v_{1_i}=\frac{\vec{v_{1_i}}\cdot\overrightarrow{M_1M_2}}
{|\overrightarrow{M_1M_2}|}=\frac{295}{\sqrt{730}}=10.9184\frac{m}{s}$.
Genauso wird $v_{2_i}$ als Geschwindigkeitsbetrag von Ball 2
\emph{in Sichtlinie} definiert:
$v_{2_i}=\frac{\vec{v_{2_i}}\cdot\overrightarrow{M_1M_2}}
{|\overrightarrow{M_1M_2}|}=\frac{-315}{\sqrt{730}}=-11.6587\frac{m}{s}$.

Indem man die Geschwindigkeitsbetr\"age
\emph{in Sichtlinie} findet, kann man eine 2D (oder 3D) Kollision als
eindimensional ansehen, wenn die eine Dimension entlang der Sichtlinie l\"auft.
Weil man einen Zwischenschritt \"uber eindimensionale Kollisionen geht,
muss man auch $v_1$ als den resultierenden Geschwindigkeitsbetrag
\emph{in Sichtlinie} (wieder: \textbf{nicht als $\mathbf{|\vec{v_1}|}$})
definieren, dasselbe gilt f\"ur $v_2$.

F\"ur den eindimensionalen Fall gelten zwei Bedingungen:

\begin{equation}
  \begin{split}
    m_1\cdot v_{1_i}+m_2\cdot v_{2_i}=
    m_1\cdot v_1+m_2\cdot v_2
    \Leftrightarrow \\
    m_1\cdot\left(v_{1_i}-v_1\right)=
    m_2\cdot\left(v_2-v_{2_i}\right)
    \quad\text{(Impulserhaltungssatz)}
    \label{eq:conservationoflinmom}
  \end{split}
\end{equation}

\begin{equation}
  \begin{split}
    \frac{1}{2}\cdot m_1\cdot v_{1_i}^2+\frac{1}{2}\cdot m_2\cdot v_{2_i}^2
    =\frac{1}{2}\cdot m_1\cdot v_1^2+\frac{1}{2}\cdot m_2\cdot v_2^2
    \Leftrightarrow \\
    m_1\cdot\left(v_{1_i}^2-v_1^2\right)=
    m_2\cdot\left(v_2^2-v_{2_i}^2\right)
    \quad\text{(Erhaltung der kinet. Energie)} 
    \label{eq:conservationofkinenergy}
  \end{split}
\end{equation}

Mit (\ref{eq:conservationoflinmom}) und
(\ref{eq:conservationofkinenergy}) habe ich ein Gleichungssystem aus zwei
Gleichungen und zwei Unbekannten (Geschwindigkeitsbetrag von Ball 1 nach
der Kollision \emph{in Sichtlinie} ($v_1$) und Geschwindigkeitsbetrag
von Ball 2 nach der Kollision \emph{in Sichtlinie} ($v_2$)):
\begin{multline*}
  m_1=m_2\frac{v_2-v_{2_i}}{v_{1_i}-v_1}
  \;\wedge\;
  m_1=m_2\frac{v_2^2-v_{2_i}^2}{v_{1_i}^2-v_1^2}
  \Leftrightarrow
  \frac{v_2-v_{2_i}}{v_{1_i}-v_1}
  =\frac{(v_2-v_{2_i})\cdot(v_2+v_{2_i})}{(v_{1_i}-v_1)\cdot(v_{1_i}+v_1)}
  \Leftrightarrow \\
  v_{1_i}+v_1=v_{2_i}+v_2
  \Leftrightarrow
  v_1(v_2)=v_2+v_{2_i}-v_{1_i}\;\wedge\;
  v_2(v_1)=v_1+v_{1_i}-v_{2_i}
\end{multline*}
dabei habe ich vorrausgesetzt, dass $v_{1_i}-v_1\not=0$,
$v_{1_i}^2-v_1^2=(v_{1_i}-v_1)\cdot(v_{1_i}+v_1)\not=0$,
$v_2-v_{2_i}\not=0$
(denn wenn die Geschwindigkeitsbetr\"age in Sichtlinie
vor- und nach der Kollision gleich w\"aren,
kann keine Kollision stattgefunden haben) und $m_2\not=0$.

setze $v_1(v_2)$ in (\ref{eq:conservationoflinmom}) ein und l\"ose nach
$v_2$ auf:

\begin{equation*}
  \begin{split}
    m_1\cdot(v_2+v_{2_i}-2v_{1_i})=m_2\cdot(v_{2_i}-v_2)
    \Leftrightarrow
    m_1v_2 +m_1v_{2_i}-2m_1v_{1_i}=m_2v_{2_i}-m_2v_2
    \Leftrightarrow \\
    v_2\cdot(m_1+m_2)=v_{2_i}\cdot(m_2-m_1)+2m_1v_{1_i}
  \end{split}
\end{equation*}

wenn man dasselbe auch f\"ur $v_1$ macht, bekommt man:
\begin{gather}
  \label{eq:vone}
  v_1(v_{1_i},v_{2_i},m_1,m_2)=
  v_{1_i}\frac{m_1-m_2}{m_1+m_2}+v_{2_i}\frac{2m_2}{m_1+m_2} \\
  \label{eq:vtwo}
  v_2(v_{1_i},v_{2_i},m_1,m_2)=
  v_{2_i}\frac{m_2-m_1}{m_1+m_2}+v_{1_i}\frac{2m_1}{m_1+m_2}
\end{gather}

Ein Spezialfall offenbart sich hier: Was ist wenn beide B\"alle
die gleiche Masse haben ? (z.B. beim Billiard, wenn man nicht-linearen
Impuls und Reibung vernachl\"assigt)
F\"ur den Fall ist $v_1=v_{2_i}$ und $v_2=v_{1_i}$.
%% ... ?

Im Beispiel oben erh\"alt man durch einsetzen in (\ref{eq:vone}): \\
$v_1=\frac{295}{\sqrt{730}}\frac{m}{s}\cdot\frac{1}{10}
-\frac{315}{\sqrt{730}}\frac{m}{s}\cdot\frac{9}{10}
=\frac{-254}{\sqrt{730}}\frac{m}{s}=-9.40096\frac{m}{s}$
und aus (\ref{eq:vtwo}) erh\"alt man: \\
$v_2=-\frac{315}{\sqrt{730}}\frac{m}{s}\cdot-\frac{1}{10}+
\frac{295}{\sqrt{730}}\frac{m}{s}\cdot 1.1
=\frac{356}{\sqrt{730}}\frac{m}{s}=13.1762\frac{m}{s}$
(negative Vorzeichen sind hier okay, da $v_1$ und $v_2$ ja nur
den Geschwindigkeitsbetrag in Sichtlinie angeben).

Im 1D-Fall w\"are ich jetzt schon fertig, $v_1$ w\"are die resultierende
Geschwindigkeit von Ball 1 und $v_2$ die von Ball 2. Das ist aber nur
1D-Fall so, weil der 1D-Fall in dieser Hinsicht einen Sonderfall darstellt
(siehe unten); Im 2- oder 3-Dimensionalen Fall muss man
zun\"achst die \"Anderung des Geschwindigkeitsbetrages in Sichtlinie
berechnen, diese dann auf den Sichtlinien-Vektor
($\overrightarrow{M_1M_2}$) projizieren und das Ergebnis zum urspr\"unglichen
Geschwindigkeitsvektor ($\vec{v_{1_i}}$ bzw. $\vec{v_{2_i}}$) addieren
um den Geschwindigkeitsvektor nach der Kollision ($\vec{v_1}$ bzw $\vec{v_2}$)
zu erhalten (beim 1-dimensionalen Fall entf\"allt dieser Schritt, weil
$v_{1_i}+\triangle v_1=v_{1_i}+v_1-v_{1_i}=v_1$).
Der Normalenvektor von $\overrightarrow{M_1M_2}$ ist
$\overrightarrow{{M_1M_2}^0}=
\cvectortwo{21\cdot {730}^{-0.5}}{17\cdot {730}^{-0.5}}
=\cvectortwo{0.7772}{0.6292}$.
Die \"Anderung des Geschwindigkeitsbetrages von Ball 1 projiziert
auf $\overrightarrow{M_1M_2}$ ist also:
$\overrightarrow{\triangle v_1}=\overrightarrow{{M_1M_2}^0}\cdot(v_1-v_{1_i})
=\cvectortwo{-15.79221}{-12.78494}$
und damit ist die Geschwindigkeit von Ball 1 nach der Kollision: \\
$\vec{v_1}=\vec{v_{1_i}}+\overrightarrow{\triangle v_1}
=\cvectortwo{-5.79221}{-7.78494}\frac{m}{s}$.
Genauso ist 
$\overrightarrow{\triangle v_2}=\overrightarrow{{M_1M_2}^0}\cdot(v_2-v_{2_i})
=\cvectortwo{19.30168}{15.62612}$
und $\vec{v_2}=\vec{v_{2_i}}+\overrightarrow{\triangle v_2}
=\cvectortwo{4.30168}{15.62612}\frac{m}{s}$.

siehe http://www.mcasco.com/p1lmc.html f\"ur mehr Erkl\"arungen \"uber
die Physik.

\end{document}